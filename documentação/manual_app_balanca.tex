\documentclass[12pt,a4paper]{article}
\usepackage[utf8]{inputenc}
\usepackage[T1]{fontenc}
\usepackage[portuguese]{babel}
\usepackage{geometry}
\geometry{margin=2.5cm}
\usepackage{graphicx}
\usepackage{hyperref}
\usepackage{enumitem}

\title{Manual do Usuário - SISAQUI \\ Sistema de Aquisição Universal para Balanças \\ Laboratório de Medições Mecânicas (LMM) - IPT}
\author{Gustavo dos Santos Ribeiro \\ Victor Nascimento Pereira}
\date{Janeiro de 2026}

\begin{document}

\maketitle

\tableofcontents
\newpage

\section{Introdução}

O SISAQUI (Sistema de Aquisição Universal) é uma aplicação desenvolvida especificamente para o Laboratório de Medições Mecânicas (LMM) do Instituto de Pesquisas Tecnológicas (IPT), com o objetivo de automatizar a coleta de dados de balanças Sartorius via conexão serial. Este software facilita medições metrológicas precisas, salvando dados em arquivos CSV para análise posterior.

Desenvolvido por Gustavo dos Santos Ribeiro e Victor Nascimento Pereira, o SISAQUI oferece uma interface intuitiva para usuários finais e um código modular para desenvolvedores, garantindo eficiência e confiabilidade nas operações laboratoriais.

\part{Manual para Usuário Final}

\section{Instalação para Usuários Finais}
Para usuários finais do laboratório, o SISAQUI é distribuído como um executável standalone, não requerendo instalação de dependências adicionais.

\subsection{Requisitos do Sistema}

\begin{itemize}
\item Sistema Operacional: Windows 10 ou superior
\item Espaço em disco: Mínimo 50 MB
\item Porta serial disponível (COM) e driver do cabo sendo usado para conexão com a balança Sartorius
\end{itemize}

\subsection{Instalação}
\begin{enumerate}
\item Acesse a pasta de rede "automatização de balanças".
\item Abra o executável \texttt{app\_balanca.exe}.
\item Execute o arquivo diretamente. Não é necessária instalação adicional.
\end{enumerate}

\textbf{Nota:} Certifique-se de que o antivírus não bloqueie a execução do arquivo. Se necessário, adicione uma exceção para o executável.

\section{Uso do Aplicativo}

\subsection{Inicialização}
\begin{enumerate}
\item Ligue a balança Sartorius e aguarde de 30 segundos a 1 minuto para que ela estabilize completamente.
\item Conecte o cabo serial (USB) da balança ao computador.
\item Abra o aplicativo SISAQUI executando o arquivo \texttt{app\_balanca.exe}.
\item Selecione a porta COM correspondente à balança no menu suspenso.
\item Clique em "Conectar" para estabelecer a comunicação serial.
\end{enumerate}

\subsection{Interface do Aplicativo}
A interface principal do SISAQUI é dividida em painéis intuitivos, facilitando o uso por usuários do laboratório.

\begin{figure}[h]
\centering
\includegraphics[width=0.8\textwidth]{figuras/interface_principal.png}
\caption{Interface principal do SISAQUI mostrando os painéis de arquivo, conexão, display e ações.}
\label{fig:interface}
\end{figure}

\subsection{Configuração do Arquivo de Dados}
\begin{itemize}
\item No campo "Ensaio:", insira o nome do arquivo onde você quer que os dados sejam salvos (ex.: \texttt{ensaio\_metrologia}).
\item Os dados serão salvos automaticamente na subpasta \texttt{dados coletados/} no arquivo com o nome escolhido (ex.: \texttt{ensaio\_metrologia}).
\end{itemize}

\subsection{Leitura e Salvamento de Dados}
\begin{itemize}
\item Após conectar, a leitura em tempo real será exibida no painel central.
\item Use os botões "PADRÃO (A)" ou "CLIENTE (B)" para salvar medições.
\item Atalhos de teclado:
  \begin{itemize}
  \item \texttt{A} ou \texttt{a}: Salvar como Padrão (A)
  \item \texttt{B} ou \texttt{b}: Salvar como Cliente (B)
  \item \texttt{Espaço}: Salvar como Genérico
  \end{itemize}
\end{itemize}

\begin{figure}[h]
\centering
\includegraphics[width=0.8\textwidth]{figuras/salvamento_dados.png}
\caption{Demonstração do salvamento de dados com botões destacados.}
\label{fig:salvamento}
\end{figure}

Note que ao salvar uma medição o painel de log do aplicativo indica que esta foi salva e qual seu tipo (Padrão, Cliente ou Genérico).

\subsection{Comandos Adicionais}
\begin{itemize}
\item Botão "ZERAR": Tara a balança (envia comando de tara).
\item Botão "Excel": Abre o arquivo CSV no Microsoft Excel (se instalado).
\end{itemize}

\subsection{Encerramento}
\begin{itemize}
\item Clique em "Desconectar" antes de fechar o aplicativo para liberar a porta serial.
\item O aplicativo fecha automaticamente a conexão serial ao ser fechado.
\end{itemize}
Jamais puxe o cabo serial enquanto o aplicativo estiver conectado, para evitar erros de comunicação, sempre clique em "Desconectar" antes de mexer no cabo.

\section{Troubleshooting para Usuários Finais}

\subsection{Erro de Conexão Serial}
\begin{itemize}
\item Verifique se a balança está ligada e conectada corretamente à porta COM selecionada.
\item Certifique-se de que nenhum outro programa está usando a porta serial.
\item Reinicie o aplicativo e tente conectar novamente.
\end{itemize}

\subsection{Erro 30 - Porta Serial Ocupada}
O erro 30 indica que a porta serial está bloqueada. Siga estes passos:

\begin{enumerate}
    \item \textbf{Desligue a balança:} Desconecte a energia da balança.
    \item \textbf{Remova o cabo serial:} Desconecte o cabo USB/serial da balança e do computador.
    \item \textbf{Espere 30 segundos:} Aguarde pelo menos 30 segundos.
    \item \textbf{Conecte apenas a alimentação:} Reconecte o cabo de alimentação da balança (sem o cabo serial).
    \item \textbf{Ligue a balança:} Ligue a balança e aguarde estabilização.
    \item \textbf{Espere 1 minuto:} Aguarde 1 minuto completo.
    \item \textbf{Conecte o cabo serial:} Agora conecte o cabo USB/serial.
    \item \textbf{Reinicie o aplicativo:} Abra o SISAQUI novamente e tente conectar.
\end{enumerate}

\subsection{Arquivo CSV Não Abre}
\begin{itemize}
\item Certifique-se de que o Microsoft Excel ou um leitor de CSV compatível está instalado.
\item Feche o arquivo se estiver aberto em outro programa.
\item Verifique se o arquivo foi criado na subpasta \texttt{dados coletados/}.
\end{itemize}

\subsection{Problemas Gerais}
\begin{itemize}
\item Consulte o painel de log inferior para mensagens de erro, bem como o painel da balança.
\item Reinicie o computador se os problemas persistirem.
\item Entre em contato com a equipe do LMM para suporte.
\end{itemize}

\part{Manual para Desenvolvedores}

\section{Instalação para Desenvolvedores}
Para desenvolvimento e modificações no código, é necessário configurar o ambiente Python.

\subsection{Requisitos do Sistema}
\begin{itemize}
\item Python 3.8 ou superior
\item Git para controle de versão
\item Editor de código (recomendado: Visual Studio Code)
\end{itemize}

\subsection{Instalação do Ambiente}
\begin{enumerate}
\item Clone o repositório:
   \begin{verbatim}
   git clone https://github.com/gustaribeiro8/leitor-balanca-sartorius.git
   cd automatizacao_balanca
   \end{verbatim}

\item Crie um ambiente virtual (recomendado):
   \begin{verbatim}
   python -m venv .venv
   \end{verbatim}

\item Ative o ambiente virtual:
   \begin{itemize}
   \item No Windows: \texttt{.venv\textbackslash Scripts\textbackslash activate}
   \item No Linux/Mac: \texttt{source .venv/bin/activate}
   \end{itemize}

\item Instale as dependências:
   \begin{verbatim}
   pip install -r requirements.txt
   \end{verbatim}

\item Execute o aplicativo em modo desenvolvimento:
   \begin{verbatim}
   python app_balanca.py
   \end{verbatim}
\end{enumerate}

\subsection{Geração do Executável}
Para criar um executável standalone:
\begin{verbatim}
python -m PyInstaller --noconsole --onefile --clean --icon=icone_sartorius.ico --add-data "icone_sartorius.ico;." app_balanca.py
\end{verbatim}

O executável será gerado na pasta \texttt{dist/}.

\section{Estrutura do Código}

\subsection{Principais Arquivos}
\begin{itemize}
\item \texttt{app\_balanca.py}: Arquivo principal da aplicação.
\item \texttt{leitura\_balanca.py}: Módulo para comunicação serial (se separado).
\item \texttt{icone\_sartorius.ico}: Ícone da aplicação.
\item \texttt{.gitignore}: Arquivo para ignorar arquivos no Git.
\end{itemize}

\subsection{Classe Principal: AppBalanca}
A classe \texttt{AppBalanca} herda de \texttt{ctk.CTk} e gerencia a interface gráfica.

\subsubsection{Métodos Principais}
\begin{itemize}
\item \texttt{\_\_init\_\_()}: Inicializa a interface e configurações.
\item \texttt{criar\_interface()}: Cria os elementos da GUI.
\item \texttt{alternar\_conexao()}: Conecta/desconecta da balança.
\item \texttt{salvar\_medida()}: Salva medições no CSV.
\item \texttt{thread\_monitoramento()}: Thread para leitura contínua.
\end{itemize}

\subsection{Comunicação Serial}
\begin{itemize}
\item Usa a biblioteca \texttt{pyserial} para comunicação.
\item Configuração: 1200 baud, 7 bits, paridade ímpar, 1 stop bit.
\item Comando de leitura: \texttt{\textbackslash x1bP\textbackslash r\textbackslash n}
\item Comando de tara: \texttt{\textbackslash x1bf4\_\textbackslash r\textbackslash n}
\end{itemize}

\section{Desenvolvimento e Modificações}

\subsection{Adicionando Novas Funcionalidades}
\begin{enumerate}
\item Analise os requisitos da nova funcionalidade e planeje as modificações necessárias.
\item Modifique a classe \texttt{AppBalanca} ou crie novos métodos conforme necessário.
\item Teste as alterações executando o script em modo desenvolvimento.
\item Verifique se a nova funcionalidade não quebra funcionalidades existentes.
\item Atualize a documentação, incluindo este manual, se aplicável.
\item Commite as mudanças no repositório Git com uma mensagem descritiva.
\end{enumerate}

\subsection{Debugging}
\begin{itemize}
\item Use o painel de log para depurar.
\item Adicione prints temporários para variáveis críticas.
\item Verifique a porta serial com ferramentas como PuTTY.
\end{itemize}

\subsection{Versionamento}
\begin{itemize}
\item Use Git para controle de versão.
\item Commite mudanças regularmente.
\item Ignore arquivos temporários com \texttt{.gitignore}.
\end{itemize}

\section{Troubleshooting para Desenvolvedores}

\subsection{Erros de Importação}
\begin{itemize}
\item Certifique-se de que todas as dependências estão instaladas.
\item Verifique a versão do Python (3.8+).
\end{itemize}

\subsection{Problemas de Serial}
\begin{itemize}
\item Teste a conexão serial com um terminal serial simples.
\item Verifique se a balança responde aos comandos.
\end{itemize}

\subsection{Erros de Build}
\begin{itemize}
\item Limpe as pastas \texttt{build/} e \texttt{dist/} antes de rebuildar.
\item Execute como administrador se houver permissões.
\end{itemize}

\subsection{Contribuição}
\begin{itemize}
\item Siga as boas práticas de código Python.
\item Documente funções e classes.
\item Teste em diferentes ambientes.
\end{itemize}

\end{document}